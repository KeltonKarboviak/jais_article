% !TEX root =  ../main.tex

\chapter{Conclusion} \label{ch:conclusion}
	
	This thesis presented the \toolname, an application designed to augment the existing features of the National General Aviation Flight Information Database (NGAFID).  The purpose of creating the application is to provide student pilots and Certified Flight Instructors (CFI) with metrics-based feedback on flight performance during critical phases of flight.  The desired effects of this are \textit{(i)} target different student learning techniques, \textit{(ii)} improve the efficiency and reduce the cost of flight training, and \textit{(iii)} reduce General Aviation (GA) accident and fatality rates since GA is the most dangerous branch of Civil Aviation.  Additionally, the application is currently geared towards analyzing the approach and landing phases as these phases of flight are where a majority of pilot-related accidents occur.
	
	Using flight data recorder (FDR) data generated by a Garmin G1000 from Cessna C172S aircraft; the application can detect safety exceedances for indicated airspeed, vertical speed, cross track, and heading during the approach phase as well as classify the result of each approach as a full-stop landing, touch-and-go landing, or a go-around during the landing phase.  For the event-driven approach to successfully characterize the safety of an approach, the safety exceedance definitions needed to be internally consistent (\ie, the parameter limits need to correspond to the same level of risk to the pilot).  In this research, the safety exceedances were re-defined in a way that makes them more consistent by an aviation statistics expert who used the distributions of parameter values found during the initial experiments.  These new definitions were then used in the newly created grading system for the purpose of scoring the pilot's flight performance based on any exceedances found during the approach analysis stage.
	
	Several new web tools were created which were integrated into the NGAFID.  These tools include interfaces for \textit{(i)} displaying histograms of the aggregated parameter values during approach phases within a chosen time-frame, \textit{(ii)} visualizing final turn phases for a single flight or an aggregate of flights at a chosen runway, and \textit{(iii)} displaying a histogram of self-defined glide path angles within a chosen time-frame at a given runway.
	
	The performance of the application run in parallel averaged 0.569 seconds per flight for the sample set of 100 flights, while an analysis of 5,923 flights averaged 0.457 seconds per flight.  This shows that the application has a reasonably short run-time and can be used practically in the NGAFID's production environment.


\section{Future Work} \label{sec:future_work}

	This research has provided many avenues for further work and refinement.  First, the greatest constraint on the accuracy of the application is the accuracy of the instrument recording the flight data, whether that be a traditional FDR or a smartphone.  This means that if data is recorded inaccurately, it is useless to the application and cannot be recovered.  For example, in several of the sample flights, the first 10 to 20 rows of data can have missing and/or spurious values due to the aircraft's sensors calibrating after first starting the FDR.  Invalid data rows can occur during the middle of a flight as well; not only when the FDR is initially turned on.  Thus, further work into filtering, sanitizing, or normalizing faulty data would be very beneficial.
	
	Second, it would be beneficial to make the algorithms more modular in order to analyze data from different sources.  For example, a source with a limited number of parameters it records, such as a smartphone, or a completely different FDR brand.
	
	Third, this research focused solely on the analysis of flight performance and generation of metrics describing the performance, thus further work in the area of UI/UX would be a great next step.  This future research would ideally focus on the most effective way to display the metrics to the user and improve upon the user-friendliness of the web interfaces introduced in this work.
	
	Lastly, the algorithms introduced in this research can be extended to analyze other phases of flight (\eg, takeoff, climb, cruise, etc.).  This means that new risk level values would need to be defined as well to fit the data found in the new analyses.
    
    Once the \toolname\ is fully integrated into the NGAFID, it will provide even more possibilities for data visualization and be easily accessible for both novice and experienced pilots.  This will allow pilots on an individual or organizational level to become more aware of bad flight habits so they may correct them in future flights and help make General Aviation safer.